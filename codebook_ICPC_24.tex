\documentclass[a4paper,10pt,twocolumn,oneside]{article}
\setlength{\columnsep}{10pt}                                                              %兩欄模式的間距
\setlength{\columnseprule}{0pt}                                                                %兩欄模式間格線粗細

\usepackage{amsthm}								%定義,例題
\usepackage{amssymb}
%\usepackage[margin=2cm]{geometry}
\usepackage{fontspec}								%設定字體
\usepackage{color}
\usepackage[x11names]{xcolor}
\usepackage{listings}								%顯示code用的
%\usepackage[Glenn]{fncychap}						%排版,頁面模板
\usepackage{fancyhdr}								%設定頁首頁尾
\usepackage{graphicx}								%Graphic
\usepackage{enumerate}
\usepackage{titlesec}
\usepackage{amsmath}
\usepackage{tikz}
\usepackage{xpatch}
\usepackage[CheckSingle, CJKmath]{xeCJK}
% \usepackage{CJKulem}
%\usepackage[T1]{fontenc}
\titlespacing{\section}{0cm}{0cm}{0cm}
\titlespacing{\subsection}{0cm}{0cm}{0cm}
\usepackage{amsmath, courier, listings, fancyhdr, graphicx}
\topmargin=0pt
\headsep=5pt
\textheight=780pt
\footskip=4pt
\voffset=-40pt
\textwidth=545pt
\marginparsep=0pt
\marginparwidth=0pt
\marginparpush=0pt
\oddsidemargin=0pt
\evensidemargin=0pt
\hoffset=-42pt

%\renewcommand\listfigurename{圖目錄}
%\renewcommand\listtablename{表目錄} 

%%%%%%%%%%%%%%%%%%%%%%%%%%%%%

\setmainfont [				%主要字型
    Path = .fonts/ttf/,
    UprightFont = *-Regular,
    BoldFont = *-Bold,
    ItalicFont = *-Italic
  ] {Consolas}

\setmonofont [        
    Path = .fonts/ttf/,
    UprightFont = *-Regular
  ] {Monaco}

\setCJKmainfont[
    Path = .fonts/ttf/,
    UprightFont = *-Regular
] {NotoSansTC}

\XeTeXlinebreaklocale "zh"						%中文自動換行
\XeTeXlinebreakskip = 0pt plus 1pt				%設定段落之間的距離
\setcounter{secnumdepth}{3}						%目錄顯示第三層

%%%%%%%%%%%%%%%%%%%%%%%%%%%%%

\makeatletter
\lst@CCPutMacro\lst@ProcessOther {"2D}{\lst@ttfamily{-{}}{-{}}}
\@empty\z@\@empty
\makeatother
\lstset{											% Code顯示
language=C++,										% the language of the code
basicstyle=\footnotesize\ttfamily, 						% the size of the fonts that are used for the code
%numbers=left,										% where to put the line-numbers
numberstyle=\footnotesize,						% the size of the fonts that are used for the line-numbers
stepnumber=1,										% the step between two line-numbers. If it's 1, each line  will be numbered
numbersep=5pt,										% how far the line-numbers are from the code
backgroundcolor=\color{white},					% choose the background color. You must add \usepackage{color}
showspaces=false,									% show spaces adding particular underscores
showstringspaces=false,							% underline spaces within strings
showtabs=false,									% show tabs within strings adding particular underscores
frame=false,											% adds a frame around the code
tabsize=2,											% sets default tabsize to 2 spaces
captionpos=b,										% sets the caption-position to bottom
breaklines=true,									% sets automatic line breaking
breakatwhitespace=false,							% sets if automatic breaks should only happen at whitespace
escapeinside={\%*}{*)},							% if you want to add a comment within your code
morekeywords={*},									% if you want to add more keywords to the set
keywordstyle=\bfseries\color{Blue1},
commentstyle=\itshape\color{Red4},
stringstyle=\itshape\color{Green4},
}

%%%%%%%%%%%%%%%%%%%%%%%%%%%%%

\begin{document}

\pagestyle{fancy}
\fancyfoot{}
%\fancyfoot[R]{\includegraphics[width=20pt]{ironwood.jpg}}
\fancyhead[L]{National Ocean University Rigby}
\fancyhead[C]{2024 Ocean Cup}
\fancyhead[R]{\thepage}
\renewcommand{\headrulewidth}{0.4pt}
\renewcommand{\contentsname}{Contents} 

\scriptsize
\tableofcontents
%%%%%%%%%%%%%%%%%%%%%%%%%%%%%

\section{計算幾何}

\subsection{基本儲存}
\lstinputlisting{Computational Geometry/basic_store.cpp}

\subsection{距離}
歐基里德距離
\[ d(p_1, p_2) = \sqrt{(x_1 - x_2)^2 + (y_1 - y_2)^2} \]
曼哈頓距離
\[ d(p_1, p_2) = \vert{x_1 - x_2}\vert + \vert{y_1 - y_2}\vert \]
\subsection{內積、外積}
\[ \vec{v_1} \cdot \vec{v_2} = {x_1}{x_2} + {y_1}{y_2} \]
\[ \vec{v_1} \times \vec{v_2} = {x_1}{y_2} - {x_2}{y_1} \]
\subsection{多邊形面積}
\[ \frac{1}{2}\vert\sum\limits_{i=1}^n \overrightarrow{OP_i} \times \overrightarrow{OP_{i+1}}\vert \]
\subsection{點與線段距離}
看點與線段兩端內積,若為負數則說明角度大於90度
則距離為點到該端點的距離
若內積皆>=0,則距離為三角形面積/線段
\subsection{判斷點是否在線段上}
\lstinputlisting{Computational Geometry/three_points_on_line.cpp}

\subsection{線段相交、交點}
\lstinputlisting{Computational Geometry/line_intersect.cpp}

\subsection{點在多邊形內部}
射線法:若點在多邊形內,則隨機選一個方向的射線出現會碰到奇數次邊
而如果碰到多邊形的點,如果射線碰到多邊形的點則重選
(需要特判點是否在多邊形的邊或頂點上)

\subsection{凸包}
\lstinputlisting{Computational Geometry/convex_hull.cpp}
\subsection{凸包技巧}
\lstinputlisting{Computational Geometry/convex_hull_trick.cpp}

\subsection{旋轉卡尺-最遠點對}
\lstinputlisting{Computational Geometry/farthest_pair.cpp}

\subsection{圓覆蓋面積}
\lstinputlisting{Computational Geometry/covered_circle.cpp}
\subsection{多邊形聯集面積}
\lstinputlisting{Computational Geometry/PolygonUnion.cpp}
\subsection{多邊形覆蓋面積}
\lstinputlisting{Computational Geometry/PolygonCover.cpp}

\subsection{極角排序}
\lstinputlisting{Computational Geometry/angle_sort.cpp}

\subsection{皮克定理(多邊形內整數點數量)}
\[ A = i + \frac{b}{2} - 1 \]
A:多邊形面積
i:內部整數點個數
b:線上整數點個數

\subsection{三分搜-最小包覆圓}
平面上給 n 個點,求出半徑最小的圓要包住所有的點。
求出圓心位置與與最小半徑。複雜度($N{\log}^2{N}$)
\lstinputlisting{Computational Geometry/three_search.cpp}

\subsection{旋轉矩陣、鏡射矩陣}
逆時針轉$\theta$角
\[\begin{bmatrix}
  cos\theta&-sin\theta  \\
  sin\theta &cos\theta 
\end{bmatrix}\]
對與x軸正向夾角為$\theta$的直線L鏡射
\[\begin{bmatrix}
  cos2\theta & sin2\theta  \\
  sin2\theta & -cos2\theta 
\end{bmatrix}\]

\section{資料結構}
\subsection{離散化}
\lstinputlisting{DSA/discretization.cpp}
\subsection{線段樹}
\lstinputlisting{DSA/segment_tree.cpp}
\subsection{莫隊算法}
\lstinputlisting{DSA/Mo's_algo.cpp}
\subsection{CDQ分治}
\lstinputlisting{DSA/CDQ.cpp}
\subsection{可持久化線段樹}
\lstinputlisting{DSA/persistent_segment_tree.cpp}

\section{動態規劃}
\subsection{0/1背包}
$O(NW)$
\lstinputlisting{dp/01knapsack.cpp}
\subsection{無限背包}
$O(NW)$
\lstinputlisting{dp/infinite_knapsack.cpp}
\subsection{有限背包}
$O(NW\log{k})$
\lstinputlisting{dp/finite_knapsack.cpp}


\section{數學}
\subsection{階乘與模逆元}
\lstinputlisting{math/factorial.cpp}
\subsection{擴展歐基里德}
\lstinputlisting{math/exgcd.cpp}
\subsection{中國剩餘定理}
\lstinputlisting{math/CRT.cpp}
\subsection{進制轉換}
\lstinputlisting{math/representation_change.cpp}
\subsection{O(1)mul}
\lstinputlisting{math/O(1)mul.cpp}
\subsection{Miller Rabin}
\lstinputlisting{math/Miller_Rabin.cpp}
\subsection{Pollard Rho}
\lstinputlisting{math/Pollard_Rho.cpp}
\subsection{FFT}
\lstinputlisting{math/FFT.cpp}
\subsection{約瑟夫問題}
\lstinputlisting{math/Josephus.cpp}
\subsection{快速求歐拉函數}
\lstinputlisting{math/quick_eular.cpp}
\subsection{矩陣}
\lstinputlisting{math/Matrix.cpp}


\section{樹論}
\subsection{LCA}
\lstinputlisting{tree/lca.cpp}
\subsection{換根DP}
\lstinputlisting{tree/change_root.cpp}
\subsection{樹哈希}
\lstinputlisting{tree/tree_hash.cpp}
\subsection{重鏈剖分}
\lstinputlisting{tree/Heavy_Link_cut.cpp}
\subsection{虛樹}
\lstinputlisting{tree/Virtual_Tree.cpp}

\section{圖論}
\subsection{最短路徑}
dijkstra $O({V}^2 + E)$
\lstinputlisting{graph/short_path/dijkstra.cpp}
floyd-warshall $O({N}^3)$
\lstinputlisting{graph/short_path/floyd_warshall.cpp}
\subsection{歐拉回路、漢米爾頓路徑}
\lstinputlisting{graph/Eular_cycle.cpp}
\lstinputlisting{graph/Hamiltonian_path.cpp}
\subsection{點雙連通分量}
\lstinputlisting{graph/BCC.cpp}
\subsection{強連通分量}
\lstinputlisting{graph/SCC.cpp}
\subsection{2-SAT}
\lstinputlisting{graph/2-SAT.cpp}

\section{字串}
\subsection{KMP}
\lstinputlisting{string/KMP.cpp}
\subsection{Hash}
\lstinputlisting{string/Double_Hash.cpp}
\subsection{minRotation}
\lstinputlisting{string/minRotation.cpp}
\subsection{Suffix Array}
\lstinputlisting{string/SA.cpp}
\subsection{馬拉車}
\lstinputlisting{string/Manacher.cpp}
\subsection{Zvalue}
\lstinputlisting{string/Zvalue.cpp}
\subsection{字典樹}
\lstinputlisting{string/Trie.cpp}
\subsection{回文樹}
\lstinputlisting{string/palTree.cpp}


\section{網路流}
\subsection{Dinic}
\lstinputlisting{flow/dinic.cpp}
\subsection{最小花費最大流}
\lstinputlisting{flow/Min-cost_Max-flow.cpp}
\subsection{最小割}
\lstinputlisting{flow/MinCut.cpp}
\subsection{匈牙利演算法}
\lstinputlisting{flow/Hungarian_Algo.cpp}
\subsection{二分圖最大權完美}
\lstinputlisting{flow/KM.cpp}


% \section{一些題目}
% \subsection{最大子矩形-玉蟾宮}
% \lstinputlisting{problem/max_submatrix.cpp}
% \subsection{次小生成樹}
% 先求出 MST 之後,窮舉每條不在 MST 上的邊放上去,而會形成環
% 把環上除了新的邊中權重最大的邊移除,判斷是否為答案
% \subsection{最大網路流}
% \lstinputlisting{problem/max_flow_example.cpp}


\section{小技巧}
\subsection{快讀/快寫}
\lstinputlisting{skill/Fast_IO.cpp}
% \subsection{progama優化}
% \lstinputlisting{skill/Progama.cpp}
\subsection{隨機數}
\lstinputlisting{skill/Random.cpp}
\subsection{Linus指令}
\lstinputlisting{skill/Linus_cmd.cpp}
\subsection{Windows對拍}
\lstinputlisting{skill/debugScript.cpp}


% \section{其他}
% \lstinputlisting{other/funny.cpp}
% \lstinputlisting{other/IwantAC.cpp}

\end{document}